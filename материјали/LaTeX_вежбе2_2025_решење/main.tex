\documentclass[a4paper, 10pt]{article}
\usepackage[margin=1in]{geometry}
\usepackage{graphicx}
%\usepackage[serbian]{babel}
\usepackage[T2A]{fontenc}
\usepackage{hyperref, amsmath}

\renewcommand{\figurename}{\textit{Слика}}
\renewcommand{\tablename}{\textit{Табела}}
\renewcommand{\contentsname}{Садржај}
\renewcommand{\listfigurename}{Списак слика}
\renewcommand{\listtablename}{Списак табела}

\title{\textbf{Друга вежба}}
\author{Јован Јовановић 627/22}
\date{}

\begin{document}
	\maketitle
	\tableofcontents
	\listoftables
	\listoffigures
	
	\pagenumbering{gobble}

\clearpage
\section{Чебишевљеви полиноми}

\pagenumbering{arabic}

\noindent Квадрат амплитудске карактеристике Чебишевљевог филтра прве врсте је облика
	\begin{equation}
		\left|H\left(j\Omega\right)\right|^2=\frac{1}{1+\epsilon^2T^2_N\left(\Omega\right)},
	\end{equation}
где је са $T_N\left(\omega\right)$ означен Чебишевљев полином прве врсте $N$-тог реда дефинисан релацијом
\[
T_N\left(\Omega\right)=\cosh\left(N\operatorname{arccosh}\Omega\right).
\]

Неколико Чебишевљевих полинома дато су у табели~\ref{Tab:cheb}, док је график функције $T_7\left(x\right)$ за вредност аргумента $x\in\left(-1.05,\,1.05\right)$ приказан на слици~\ref{Fig:cheb}.
\begin{table}[!h]
	\centering
	\caption{Чебишевљеви полиноми}
	\begin{tabular}{c|c}
		$N$ & $T_N\left(x\right)$\\
		\hline
		0 & 1\\
		1 & $x$\\
		2 & $2x^2 - 1$\\
		3 & $4x^3-3x$\\
		4 & $8x^4-8x^2+1$\\
		7 & $64x^7-112x^5+56x^3-7x$
	\end{tabular}
	\label{Tab:cheb}
\end{table}

\begin{figure}[!h]
	\centering
	\includegraphics{pgfplots_fig.pdf}
	\caption{Чебишевљев полином прве врсте седмог реда.}
	\label{Fig:cheb}
\end{figure}

\clearpage

\section{Пакет \texttt{circuitikz}}
Пасивна мрежа приказана је на слици~\ref{Fig:circuitikz}.
\begin{figure}[!h]
	\centering
	\includegraphics{circuitikz_fig.pdf}
	\caption{Пасивна мрежа}
	\label{Fig:circuitikz}
\end{figure}

\end{document}
