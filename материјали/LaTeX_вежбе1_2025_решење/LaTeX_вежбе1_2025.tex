\documentclass[10pt]{article}
\usepackage[margin=1in]{geometry}
\usepackage[T2A]{fontenc}
\usepackage{datetime}
\usepackage[serbian]{babel}

\begin{document}

    \pagestyle{empty}

    \noindent Пошто је освануо први снежни дан ове јесени (мада тренутна температура од 2\textdegree C више одговара зимској него јесењој), започећемо једним прикладним и лепо форматираним цитатом у окружењу \verb|quotation|:

    \begin{quotation}
    \quotedblbase Немојте бити много пословни и практични и из тог угла нападати и оговарати снег. Снег је велика небеска лепота и доброта. Снег треба волети, треба му се радовати, треба га поштовати.\textquotedblleft

    \hfill --- Душко Радовић, \textit{Баш свашта, Сабрани списи}, 2006.
    \end{quotation}

    \textnormal{U tekstu se koriste} \textit{kurzivna}, \textbf{podebljana}, \textbf{\textit{kurzivna podebljana slova}}, {\large \textbf{\textit{krupna kurzivna podebljana slova}}} i slova \textsf{sans serif} familije.

    \vspace{10pt}

    \begin{center}
    У наставку је дата функција за израчунавање вредности факторијела
    \end{center}

    \begin{verbatim}
        def fact(n):
            if n == 0:
                return 1
            else:
                return n*fact(n - 1)
    \end{verbatim}

    \vspace{5pt}
    \begin{tabular}{||l||r||}
        \hline
        Ime i prezime &	Broj indeksa\\
        \hline
        Petar Petrovi\'c & 621/23\\
        Jovan Jovanovi\'c & 123/23\\
        \v Zivko \v Zivkovi\'c & 15/18\\
        \hline
    \end{tabular}

    \medskip

    \begin{itemize}
        \item За приказ кода користимо \verb|verbatim| окружењe, које даје излаз
        у \verb|monospaced| фонту, у коме сваки карактер заузима исти хоризонтални размак.
        \item Окружење \verb|tabular| користимо за генерисање табеле.
        \item За подебљана слова се користи \verb|\textbf| команда,
        а за курзивна \verb|\textit|.
    \end{itemize}

    \medskip

    \begin{enumerate}
        \item Прва нумерисана ставка
        \item Друга нумерисана ставка
        \item Трећа нумерисана ставка. Ово је наставак треће ставке како
        бисмо видели како се понаша у новом реду.
    \end{enumerate}

    \medskip

    \begin{description}
        \item[itemize:] Окружење за генерисање ненумерисане листе.
        \item[enumerate:] Окружење за генерисање нумерисане листе.
        \item[description:] Окружење за генерисање описне листе.
        У овом окружењу се налазимо тренутно.
    \end{description}

    \vfill
    \noindent На \textit{\textbf{дну стране}} лево \hfill и десно\\
    треба исписати датум и време коришћењем команди \verb|\today| и \verb|\currenttime|.

    \vspace{1cm}

    \noindent \today \hfill \currenttime

\end{document}
