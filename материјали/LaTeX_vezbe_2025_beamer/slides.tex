\documentclass{beamer}
%\documentclass[handout]{beamer} % handout opcija sklanja dinamiku i vizuele efekte iy slajdova

% Koristi se uz handouts opciju:
%\usepackage{pgfpages}
%\pgfpagesuselayout{2 on 1}[a4paper, border shrink=5mm] % 2 slajda na jednu stranu
%\pgfpagesuselayout{4 on 1}[a4paper, landscape, border shrink=5mm] % 4 slajda na jednu stranu
%\pgfpagesuselayout{8 on 1}[a4paper, border shrink=2mm] % 8 slajdova na jednu stranu
%

\usetheme{default}% {Goettingen} {Copenhagen}
\setbeamercovered{transparent}%{invisible} % važno za dinamiku
\usecolortheme{default} %{seahorse}{rose}{wolverine}{infolines}

%\usefonttheme{serif} % Podešavanje serif fonta. Korisno!

\usepackage[utf8]{inputenc}
\usepackage[T1, T2A]{fontenc}
\usepackage[serbian]{babel}

\usepackage{datetime}

\title[\textbf{Naslov prezentacije}]{\textbf{\huge Naslov prezentacije}}
\author{\textbf{Petar Petrović}}
\date{\today}


\begin{document}


\begin{frame}

    \titlepage

\end{frame}


\section{Dinamika običnog teksta}


\begin{frame}{Jednostavna dinamika}

    \pause
    Prvi red \\
    \pause
    Drugi red \\
    \pause
    Treći red

\end{frame}


\begin{frame}{Nešto složenija dinamika}

    \onslide<2->{Prvi red}\\
    \onslide<3-3>{Drugi red}\\
    \onslide<2-2>{Opet prvi red, ali kratko}\\
    \onslide<4->{Treći red}\\

\end{frame}


\section{Nabrajanje stavki}


\begin{frame}{Nabrajanje}

    \begin{itemize}
        \item <2-> Prva stavka
        \item <3-> Druga stavka
        \item <4-> Treća stavka
        \item <2-> Još jedna prva stavka
    \end{itemize}

\end{frame}


\begin{frame}{Nabrajanje redom}

    \begin{itemize}
        \item <+-> Prva stavka
        \item <+-> Druga stavka
        \item <+-> Treća stavka
    \end{itemize}

\end{frame}


\section{Numerisane liste i jednačine}


\begin{frame}{Numerisana lista}

    \begin{enumerate}
        \item <2-> Prva stavka
        \item <3-> Druga stavka
        \item <4-> Treća stavka
    \end{enumerate}

\end{frame}


\begin{frame}{Progresivno boldovanje}

    \begin{enumerate}
        \item \textbf<2->{Prva stavka}
        \item \textbf<3->{Druga stavka}
        \item \textbf<4->{Treća stavka}
    \end{enumerate}

\end{frame}


\begin{frame}{Progresivno boldovanje: naglašavanje jedne stavke}

    \begin{enumerate}
        \item \textbf<2-2>{Prva stavka}
        \item \textbf<3-3>{Druga stavka}
        \item \textbf<4-4>{Treća stavka}
    \end{enumerate}

\end{frame}


\begin{frame}{Jednačine + invisible}

    \setbeamercovered{invisible}
    Pitagorina teorema glasi:

    \pause
    \bigskip

    $c^2=a^2+b^2$

    %\[c^2=a^2+b^2\]

    %\begin{equation}
    %    c^2=a^2+b^2
    %\end{equation}

    \bigskip
    \pause

    \begin{center}
        Znate li dokaz?
    \end{center}

    \setbeamercovered{dynamic} % ili {transparent}

\end{frame}


\section{Ubacivanje slika}


\begin{frame}{Slike}

    \pause

    Često je potrebno ubaciti neku sliku, kao npr.

    \bigskip
    \pause

    \includegraphics[scale=0.5]{pgfplots_fig.pdf}

    \pause

    što je već dobro poznat Čebiševljev polinom prve vrste sedmog reda.

    \bigskip
    \pause

    Ali slika nije bila zamagljena kao što smo očekivali!!!

\end{frame}


\begin{frame}{Slika + invisible}

    \setbeamercovered{invisible}%{dynamic}

    \onslide<2->{Često je potrebno ubaciti neku sliku, kao npr.}

    \bigskip
    \onslide<3->{\includegraphics[scale=0.5]{pgfplots_fig.pdf}}

    \onslide<4->{što je već dobro poznat Čebiševljev polinom prve vrste sedmog reda.}

    \bigskip
    \onslide<5->{Umesto da zamaglimo sliku koristili smo invisible format.}

    \bigskip
    \onslide<6->{Da bi se slika zamaglila potrebno je napraviti novu zamagljenu sliku (malo više posla).}

    \setbeamercovered{dynamic}

\end{frame}


\section{Boje}


\begin{frame}{Bojena slova}

    \begin{Large}
    \textcolor{red}{Crveni red} \\
    \pause
    \textcolor{blue}{Plavi red} \\
    \pause
    \textcolor{cyan}{Cyan red (svetlo plavi)} \\
    \pause
    \textcolor{magenta}{Magenta red} \\
    \pause
    \textcolor{yellow}{Žuti red} \\
    \pause
    \textcolor{green}{Zeleni red} \\
    \end{Large}
\end{frame}


\beamertemplatesolidbackgroundcolor{teal!20}

\begin{frame}{\textcolor{teal}{\textbf{Bojena pozadina}}}

    \begin{Huge}
        \textcolor{teal}{sa tamnim slovima}
    \end{Huge}

\end{frame}

\beamertemplatesolidbackgroundcolor{white}


\section{Kolone}


\begin{frame}{Raspoređivanje u dve kolone}

    \begin{columns}
        \column{5cm}
            \onslide <2-> Gore levo \\
            \onslide <3-> Dole levo \\
        \column{5cm}
            \onslide <4-> Gore desno \\
            \onslide <5-> Dole desno \\
    \end{columns}

\end{frame}


\begin{frame}{Raspoređivanje u više kolona}

    \begin{columns}
        \column{3cm}
            \onslide <2-> Gore levo \\
            \onslide <2-> Dole levo \\
        \column{3cm}
            \onslide <3-> Gore u sredini \\
            \onslide <3-> Dole u sredini \\
        \column{3cm}
            \onslide <4-> Gore desno \\
            \onslide <4-> Dole desno \\
    \end{columns}
\end{frame}

\end{document}
